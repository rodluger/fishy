\documentclass[modern]{aastex62}

% Load the corTeX style definitions
\input{cortex}

% Bibliography stuff
\bibliographystyle{aasjournal}

% Begin!
\begin{document}

% Title
\title{The Mapping Problem}

% Author list
\author[0000-0002-0296-3826]{Rodrigo Luger}
\email{rluger@flatironinstitute.org}
\affil{Center~for~Computational~Astrophysics, Flatiron~Institute, New~York, NY}
%
\author{David W. Hogg}
\affil{Center~for~Computational~Astrophysics, Flatiron~Institute, New~York, NY}

%
\section{Introduction}
%
Check out \citet{Luger2019} and stuff, and look at
Figures~\ref{fig:g_rigid} and \ref{fig:photometry_nullspace}.

%

\begin{figure}[p!]
    \begin{centering}
    \includegraphics[width=\linewidth]{figures/g_rigid.pdf}
    \oscaption{g_rigid}{%
        The Doppler basis for a rigidly rotating star
        computed up to spherical 
        harmonic degree $l=10$. Rows correspond to the degree $l$ and
        columns correspond to the order $m$. These functions encode
        the contribution of each spherical harmonic to the rotational
        broadening of features in the stellar spectrum.
        \label{fig:g_rigid}
    }
    \end{centering}
\end{figure}

\begin{figure}[p!]
    \begin{centering}
    \includegraphics[width=\linewidth]{figures/photometry_nullspace.pdf}
    \oscaption{photometry_nullspace}{%
        The photometric basis for a star rotating about an axis perpendicular
        to the line of sight. Compare to Figure~\ref{fig:g_rigid}.
        \label{fig:photometry_nullspace}
    }
    \end{centering}
\end{figure}

% Bibliography
\bibliography{bib}

\end{document}
