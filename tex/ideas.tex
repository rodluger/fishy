\section{Taylor Expanding the Doppler operator \\ (and why this is a bad idea)}
\label{sec:taylor}
%
An alternative way to tackle the problem is to linearize the Doppler 
operator via a Taylor expansion in $\alpha$ about $\alpha=0$:
%
\begin{align}
    \label{eq:taylor:I}
    I(\xi, x, y) 
        &=
        I_0(\xi_0, x, y) \Bigg|_{\alpha=0}
        + 
        \frac{\mathrm{d}I_0(\xi_0, x, y)}{\mathrm{d}\alpha} \Bigg|_{\alpha=0} 
            \Delta\alpha(x, y)
        + 
        \frac{1}{2}\frac{\mathrm{d}^2I_0(\xi_0, x, y)}{\mathrm{d}\alpha^2} 
            \Bigg|_{\alpha=0} \Delta\alpha(x, y)^2
        +
        ... 
\end{align}
%
Since the dependence of Equation~(\ref{eq:xi0}) on $\alpha$ is trivial,
the derivatives of the spectrum $I_0(\xi_0)$ with respect to
$\alpha$ are simply
%
\begin{align}
    \frac{\mathrm{d}^nI_0(\xi_0, x, y)}{\mathrm{d}\alpha^n} &=
    \dfrac{\mathrm{d}^nI_0(\xi_0, x, y)}{\mathrm{d}\xi_0^n} \nonumber\\ &\equiv
    I_0^{(n)}(\xi_0, x, y)
    \quad.
\end{align}
%
Given this result, and noting that $\xi_0 = \xi$ when $\alpha = 0$,
we may re-write Equation~(\ref{eq:taylor:I}) as
%
\begin{proof}{Taylor}
    \label{eq:taylor:ISum}
    I(\xi, x, y) 
        &=
        \sum_{n=0}^\infty
            \frac{I_{0}^{(n)}(\xi, x, y)}{n!}
            \Delta\alpha(x, y)^n
        \quad ,
\end{proof}
%
The utility of this expression is that the velocity dependence of the spectrum
is now entirely encoded in the terms $\Delta\alpha(x, y)^n$, which are
\emph{independent of wavelength}. We can further decouple the spatial
dependence from the spectral dependence by expressing the intensity field
as an expansion over spherical harmonics $Y_{lm}(x, y)$ in the sky-projected
coordinates:
%
\begin{align}
    I_0(\xi, x, y) = \sum_{l=0}^\infty\sum_{m=-l}^{l} a_{lm}(\xi) Y_{lm}(x, y)
    \quad .
\end{align}
%
Equation~(\ref{eq:taylor:ISum}) now reads
%
\begin{align}
    \label{eq:taylor:IYlm}
    I(\xi, x, y) 
        &=
        \sum_{n=0}^\infty
            \sum_{l=0}^\infty\sum_{m=-l}^{l}
                \frac{a_{lm}^{(n)}(\xi)}{n!}
                Y_{lm}(x, y)\Delta\alpha(x, y)^n
            \quad .
\end{align}
%
Finally, integrating this equation over the visible disk of 
the star, we arrive at an equation for the observed spectrum:
%
\begin{align}
    \label{eq:taylor:S}
    S(\xi) 
        &=
        \sum_{n=0}^\infty
            \sum_{l=0}^\infty\sum_{m=-l}^{l}
                \frac{a_{lm}^{(n)}(\xi)}{n!}
                \iint\limits_{\mathcal{S}(x, y)}
                Y_{lm}(x, y)\Delta\alpha(x, y)^n
                \mathrm{d}{\mathcal{S}(x, y)}
            \quad .
\end{align}
%
By expanding the spectrum in both the spectral and spatial dimensions, we
have effectively decoupled the two. However, Equation~(\ref{eq:taylor:S}) 
is impractical for three reasons. First, the surface integrals can be very
difficult to solve (although we will show later how these integrals may be
approximated analytically). Second, it requires knowledge of high order 
derivatives of the spectrum, which may not always be easy or convenient to 
compute. Third, and most important, the series expansion in $n$ is typically 
extremely slow to converge and is therefore only practically useful in the 
limit that the Doppler shift is much smaller than the typical width of a 
spectral line; see the notebook link next to Equation~(\ref{eq:taylor:ISum}) 
for an example.